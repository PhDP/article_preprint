\documentclass[letterpaper,twocolumn,superscriptaddress,showkeys,longbibliography]{revtex4-1}
\usepackage[utf8]{inputenc}
\usepackage{color,dcolumn,graphicx,hyperref}
\hypersetup
{
    colorlinks = true, linkcolor = blue, citecolor = blue, urlcolor = blue,
}

\begin{document}

\title{Developing a preprint culture in biology}

\author{Philippe Desjardins-Proulx}
\email[E-mail: ]{philippe.d.proulx@gmail.com}
\affiliation{Theoretical Ecosystem Ecology laboratory, Universit\'e du Qu\'ebec \`a Rimouski, Canada.}
\affiliation{Quebec Center for Biodiversity Science, McGill University, Canada.}
\affiliation{D\'epartement des sciences biologiques, Universit\'e du Qu\'ebec \`a Montr\'eal, Canada.}

\author{Ethan P. White}
\affiliation{Departement of Bology, Utah State University, United-States of America.}

\author{Joel J. Adamson}
\affiliation{Ecology, Evolution and Organismic Biology, University of North
Carolina at Chapel Hill, United-States of America}

\author{Timoth\'ee Poisot}
\affiliation{Theoretical Ecosystem Ecology laboratory, Universit\'e du Qu\'ebec \`a Rimouski, Canada.}
\affiliation{Quebec Center for Biodiversity Science, McGill University, Canada.}
\affiliation{International Network for Next-Generation Ecology.}

\author{Karthik Ram}
\affiliation{Environmental Science, Policy, and Management. University of California, Berkeley, United-States of America.}

\author{Dominique Gravel}
\affiliation{Theoretical Ecosystem Ecology laboratory, Universit\'e du Qu\'ebec \`a Rimouski, Canada.}
\affiliation{Quebec Center for Biodiversity Science, McGill University, Canada.}

\keywords{Publishing; Preprint servers; Green Open Access; arXiv.}

\maketitle

\section{Introduction}

Public preprint servers allow authors to make manuscripts publicly available
before, or in parallel to, submitting them to journals for traditional
peer-review. The rationale for preprint servers is fundamentally simple: to make
the results of scientific research available to the scientific community as soon
as possible, instead of waiting until the peer-review process is fully completed.
Sharing manuscripts using preprint servers has numerous advantages including: 1) rapid
dissemination of work-in-progress to a wider audience; 2) improved peer review by
encouraging feedback from the entire research community;
and 3) a fair and straightforward way to establish precedence.

Preprints began to gain popularity 20 years ago with the advent of arXiv, an open
preprint server widely used in physics and mathematics \cite{gin11}. Preprints
are also integral to the culture of other scientific fields.  Paul Krugman noted
that, in economics, the \emph{traditional model of submit, get refereed,
publish, and then people will read your work broke down a long time ago. In
fact, it had more or less fallen apart by the early 80s} \cite{kru12}. In
addition to a section in arXiv, economists have also the RePEc (Research Papers
in Economics) initiative, which aims to create an archive of working papers,
manuscripts, and book chapters.

Despite the success of this approach in other
fields, most manuscripts in biology are not posted to preprint servers and
are therefore not seen by more than a handful of other scientists prior to
publication. In this article, we: 1) highlight the
advantages of open preprint servers for both scientists and publishers; 2)
address several misconceptions about preprints that are common among biologists;
3) discuss the preprint policies of major publishers in biology;
and 4) review the most popular preprint servers currently available.

\section{The case for public preprints}

The first and most often discussed advantage of arXiv and open preprints is
speed (Figure~\ref{fig:map}). The time between submission and the official
publication of a manuscript can be measured in months or even years. During
this time, the research is known only to a select few: colleagues, editors,
reviewers.  Thus, the science cannot be used, discussed, or reviewed by the
wider scientific community.  If the science being published is a year
or more behind the cutting edge, then science will progress more slowly. While
conferences and informal sharing of ideas among research groups may ameliorate
this issue, this type of communication is inherently limited to a restricted
subset of scienctists. Posting preprints ensures that science is not
constrained by small networks of scientists exchanging ideas.  Paul Ginsparg
created arXiv.org in part for democratic reasons: he wanted everyone from students
in small universities to Ivy-League professors to have access to the most recent
scientific \emph{ideas}. Ginsparg's revolutionary idea was simply to use the power
of the internet for preprints, not just for the end product, so the scientific
process can be open to all as soon as possible.

Posting manuscripts as preprints also has the potential to result in higher
quality science, by allowing for pre-publication feedback
from a large pool of reviewers. Prepublication reviews by a small network of
colleagues are common in the biological sciences and are an important part of
the scientific process. These "friendly" reviews increases the chances of errors
being caught prior to publication, allow errors to be identified
prior to spending months in review, and allows authors to improve how they
communicate the results of their research by modifying their writing in response
to early feedback. Preprint servers offer a way to extend this network of
colleagues to the entire scientific
community, increasing the number of opportunities for review and revision
prior to publication, and resulting in higher quality submissions and publications.
As such, preprints can be seen as an
integral aspect of a vigorous peer-review process \cite{hoc12}. 

%It is unclear to me whether we should keep this full paragraph or try to distill
%it to a sentence or two at the end of the previous paragraph.
Furthermore, the formal peer-review process as a whole is critically over-loaded. As
the number of active scientists increases and the pressure to publish
increases, it has become increasingly difficult for journals to find
reviewers \cite{hoc09}.  At the same time, rejection rates are high in most journals
\cite{aar08,roh09}, and when not invited to submit a revision, authors must start the
process over again at another journal. As a result, initiatives to reduce time from submission to
publication have emerged across the scientific community. Rohr et al.
\cite{roh09} called for the recycling and reuse of peer-reviews:
by attaching previous reviews and detailed replies to a new submission, both
the editor and the referees can gauge the work done on the manuscript, and
perhaps evaluate it with less prejudice. In a similar way, the \emph{Peerage
of Science} initiative allows authors to seek anonymous pre-review by their
peers. Some journals now accept to publish papers which received good
evaluations, with \emph{Animal Biology} having recently accepted first a paper
reviewed entirely with the \emph{Peerage of Science} \cite{abb12}, effectively
outsourcing the review process. A widespread use of preprint servers can
achieve the same goal of reducing the time spent in review. By allowing
open comments and criticisms prior to submission, the authors will receive valuable
feedback and can improve the version which will be submitted. With a rich
enough community of scientists depositing preprints, and commenting on them,
the process of an open pre-review can become widespread and will overall
increase the quality of first submissions.

\begin{figure}[ht!] \centering\includegraphics[width=0.50\textwidth]{map.pdf}
\caption { It can take several months, and even a few years, before a submitted
paper is officially published and citable.  The average time to publication
varies greatly between journals and can be as low as 104 days (Evolution for
2011) to 213 (PLOS One in 2010).  Meanwhile, few people are aware of the
research that has been done since, typically, only close colleagues are given
access to the preprints. With public preprint servers, the science is
immediately available and can be openly discussed, analyzed, and integrated into
current research. It benefits both science and publishers. Both want the papers
to be well-known and cited, and public preprints make it possible to integrate
research even before publication, greatly improving immediacy.  }
\label{fig:map} \end{figure}

Finally, public preprint servers offer a fair way to establish intellectual
priority by making the work available when immediately after it is complete.
Some manuscripts will spend much more time in the review process and/or in
production after acceptance, than others. This means that
that publication and acceptance dates do not accurately characterize who
came up with an idea first. For this reason, mathematicians and physicists
have embraced arXiv in part to establish priority in a fair way \cite{gin11,cal12}.

\section{Preprints in biological sciences}

In contrast to other disciplines, the field of biology has effectively no preprint
culture, with the exception of small pockets of primarily highly quantitative
research (e.g., epidemiology). While, submitting to preprint servers has become
more common in the past few years, and the quantitative biology section in arXiv
is experiencing rapid growth in submissions \cite{cal12}, the number of biology
papers submitted to preprint servers still represents only a small fraction of
the total research produced in biology.

There are a number of reasons that biologists have not traditionally had a
culture of sharing preprints, many of which are based on common misconceptions
regarding the costs and benefits of sharing preprints. For example, in contrast
to other fields there is a perception in biology that public preprints
make it easier to steal ideas. However, there is no evidence of this happening
in the numerous other fields that have adopted preprint servers, and since preprint
servers create a clear record of who had the idea first, and when, this
appears to be a largely unfounded concern. In other fields preprints serve the
opposite role, they allow straightforward establishment of precendence, letting
researchs lay claim to an idea thus prevening it from being "stolen".

Another major concern is that if a researcher posts a public preprint that they
will not be allowed to submit their paper to their journal of choice. This is 
based on a certain interpretation
of the Ingelfinger rule: scientists should not publish the same manuscript twice
\cite{alt96}. However, preprints are not peer-reviewed papers and posting to a
public server does not constitute publication in a peer-reviewed journal.
A preprint is simply a document that allows ideas to spread and be discussed, 
but it is not yet formally validated by the peer-review system. This is why the
majority of publishers do not see arXiv and similar services as a violation of the
Ingelfinger rule. Almost all of the major publishers in biology are preprint-friendly,
including: Nature Publishing Group, PLOS, BMC, PNAS, Science (mostly) \ref{table:policies},
and all the journals from Elsevier %something is off in the formatting here
and Springer.  The Ecological Society of America and the Genetics Society of
America also recently changed their policies to allow public preprints.  Few
scientific publications will not consider a manuscript submitted to a public
preprint server. In fact, \emph{Nature} responded to the rumour that they refused
manuscripts submitted to arXiv by saying that ``\emph{Nature} never wishes to
stand in the way of communication between researchers. We seek rather to add
value for authors and the community at large in our peer review, selection and
editing'' \cite{nat05}. Still, a few journals adopt a ``by default'' hostile attitude
towards preprints, mostly due to the lack of clear policy of the publishers, or
perhaps because a preprint culture has not developed in biology and the practice
is still considered unusual. As an example, Wiley-Blackwell, which publishes
some of the leading journals in biology, has no official policy on the matter.

\ref{table:policies}.

\begin{table*}
    \centering
    \begin{tabular}{|ll|}
    \hline
    Publisher                                   & Policy \\
    \hline
    Springer                            	& Accept \\
    BMC                                 	& Accept \\
    Elsevier                            	& Accept \\
    Nature Publishing Group             	& Accept \\
    Public Library of Science           	& Accept \\
    Genetics Society of America                 & Accept \\
    Royal Society                       	& Accept \\
    National Academy of Science (USA)           & Accept \\
    Ecological Society of America       	& Accept \\
    Oxford Journals                             & Accept \\
    Science                             	& Ambiguous \\
    Wiley-Blackwell                       	& No general policy \\
    British Ecological Society                  & No answer to our query \\
    \hline
    \end{tabular}
    \caption{Policies for important publishers in biology. Some publishers
tolerate preprints except for a few of their medical journals, e.g.: Journal
of the National Cancer Institute from Oxford and The Lancer from Elsevier.}
    \label{table:policies}
\end{table*}

\section{Current offerings}

We briefly discuss the main options to submit preprints to open servers:
arXiv.org, Figshare, and the upcoming PeerJ and F1000Research.

\subsection{arXiv}

arXiv (\url{http://arxiv.org/}) is the most widely-used preprint server today,
and its use is almost universal in some branches of mathematics and physics.
arXiv provides a reliable citation system for all eprints and is especially
popular in high-energy physics. Physicist Paul Ginsparg created arXiv in 1991
for theoretical high-energy physicists to communicate preprints via email and
ftp, and soon thereafter adopted the newly created world-wide
web\cite{jackson2002preprints}.  arXiv now receives over 7 000 submissions per
month (\url{http://arxiv.org/show_monthly_submissions}) and divides its
submissions into subcategories of physics, mathematics, computer science,
quantitative biology, finance and statistics.  The quantitative biology category
includes subcategories for Populations and Evolution, Quantitative Methods and
other categories that may be of interest to biologists.

Submission to arXiv is fully automated.  Authors can submit \TeX{}/\LaTeX{}
documents that are compiled on the server or directly submit in PDF/PS format
(for example, as exported by a word processor).  A moderation system was put in
place in 2004: papers must be categorized by an endorser. At least one author of
a paper must be an endorser that has previously submitted a paper or has
received permission to submit to a particular category.  Many authors in
mathematics and physics submit papers as soon as they are ready for review by
colleagues, although another popular option is submitting simultaneously to a
journal and arXiv.

Authors must either have their arXiv submission available under an open license
or grant arXiv a non-exclusive and irrevocable license to distribute the work.
In either case, arXiv does not require copyright transfer and only requires the
rights to distribute submitted articles in perpetuity. Thus, submitting to arXiv
does not prevent the authors from transferring their rights to a
publisher.

Most papers posted to arXiv are eventually printed in journals but there are
notable exceptions including Perelman's landmark paper leading to the proof of
the Poincar\'{e} conjecture \cite{2002math.....11159P}.  However, arXiv has
never sought to replace scientific journals and explicitly states that it serves
a different function as ``an openly accessible, moderated repository for
scholarly articles in specific scientific disciplines.'' arXiv is now
administered by the Cornell University Libraries, with funding coming from
voluntary pledges by academic institutions along with matching funds from the
Simons Foundation \cite{arxiv_future}.  One-hundred twenty six of the top
two-hundred institutions in terms of downloads have provided the operating
budget for arXiv over the next five years.  This plan reduces the financial
burden on Cornell University and transfers governance to a collaborative
community in accordance with arXiv's key principles.  arXiv takes numerous
measures to ensure that the repository will remain permanently available and
submissions will be readable.

\subsection{Figshare}

Figshare (\href{http://figshare.com/}{http://figshare.com/}) is an open server
that allows scientists to submit any research output: manuscript, figures,
datasets, videos, theses, presentations, and so on. There are no rules to limit
what constitutes a research output and, unlike arXiv, there is no endorser
system. All figshare content has a unique digital object identifier (DOI) like
any journal article, thus offering a permanent and stable link to the content.
A flexible tag system is used to classify each item. All content can be
commented and is licensed under the Creative Commons (CC-BY) license, except
datasets which are published under CC0. The CC-BY license grants rights to
\emph{copy, distribute, display and perform the work and make derivative works
based on it only if they give the author or licensor the credits in the manner
specified by these.}  CC0 is the most permissive license and effectively puts
the work in public domain (no rights reserved) or, if it is not possible in the
given jurisdiction, provides a simple permissive license.

One of the biggest advantage of figshare over arXiv is that is it not limited to
quantitative sciences. arXiv.org has sections on quantitative biology but might
not be appropriate for non-quantitative work. With its flexible approach to
preprints, figshare offers an important alternative to arXiv for empirical
biologists. Furthermore, by allowing all types of content, figshare arguably
provides an archive for early results (e.g.: figures, lab presentations).

\subsection{PeerJ}

PeerJ (\href{https://peerj.com/}{https://peerj.com/}) is a new publishing system
that combines both a preprint server, and a peer reviewed journal.  It is
focused on the biological and medical sciences, which may help overcome the
perception that preprints do not have a home in biology.  PeerJ allows
commenting on posted preprints, improving the potential for pre-publication
dialog. In addition, preprints can be made private if the authors choose, and
shared only with selected colleagues. While this reduces some of the benefits of
preprints described above, it may allow some researchers who would not otherwise
post preprints to begin to explore the possibility in a manner appropriate to
their current circumstances.

In contrast to other preprint servers users cannot post unlimited public
preprints for free. One preprint per year can be posted for free and a onetime
(i.e. lifetime) fee of 99 dollars allows the posting of unlimited public
preprints. It is also worth noting that the preprint server is not tied to the
journal, so preprints can be posted regardless of where they will eventually be
submitted for publication.

PeerJ uses the CC-BY-SA 3.0 license, which is similar to the CC-BY license used
by Figshare but adds the \emph{Share-alike} (sa) restriction that derivate works
need a license identical to the license that governs the original work.

\subsection{F1000Research}

F1000Research is not a public preprint server like the previous three servers.
Whereas arXiv, Figshare, and PeerJ offer an option to submit a manuscript
without having it reviewed, papers submitted to F1000Research will eventually
be reviewed. Thus, F1000Research offers a hybrid model with publicly available
manuscripts at time of submission and standard peer-reviews. Manuscripts are
considered ``accepted'' and will only be indexed after two positive referee
response. F1000Research works closely with data providers to integrate raw
data to the paper. Forinstance, upon submitting a paper, authors are asked to
upload their data, which are then integrated in \emph{e.g.} FigShare widgets,
the DOI of which are given in the paper when the data are first mentionned.
The licensing of the data is similar to the one used by FigShare, meaning that
the articles are free to access, and can be redistributed readily. By putting
much effort in integrating data to the paper, F1000Research is working to make
science more reproducible and open.

\subsection{Github}

This manuscript was developed entirely as an open project on github. github is
one of several hosting services for collaborative development using the git
version control system (VCS).  git is a decentralized revision control system
created by Linus Torvalds and is used primarily to develop software, including
the Linux kernel. Git provides powerful features that allow numerous contributers to work asynchronously on the same project, often in parallel branches, all of which can be effortlessly merged and version controlled. Although created primarily for software development, git is ideal for academic research since it provides a way to version control and collaborate on every step of the manuscript development process, from data manipulation and analysis to writing and revision. For example, during the development of this manuscript, each author
would clone the project (\emph{i.e.} make a personal copy), modify it, and then merge their changes into a master branch.This takes the preprint process to an entirely new level, where the entire writing process is open from the beginning.
% I would avoid using the term forking since it is not a central piece of the section. It is one specific way of collaborating on a github repo when everyone does not have rights to it. People could just as easily work on a single branch, all with write access, either directly to the master branch or to individual branches that could be merged. So in other words, we should keep the language general so people don't get bogged down with trying to understand what a fork is.
% also I changed one author clones, then submits pull request langauge a bit since the main point of using git is that we can all work simultaneously unlike svn.
% I took out "These techniques are widely used in open source software development." because git can also be used for closed source software, offering the same advantages, where the repositories are private. Paid git accounts allow for private repos.

\section{Conclusion}

Open preprint servers offer a great opportunity for open science, especially if
the community embraces the idea of discussing preprints. Initiatives like
Haldane's Sieve (\href{http://haldanessieve.org/}{http://haldanessieve.org/}), a
new blog discussing arXiv papers in population genetics, will help make arXiv
attractive for scientists looking to promote their work. These initiatives are
important to fully exploit the potential of open preprints servers. Posting
preprints online increases the community of available informal peer reviewers,
and uses the internet for its original community-building purposes.

% Aren't we bridging more than cultural and geographic? I mean, we are essentially cross pollinating ideas. e.g. It took a lot time for MCMC to become popular in ecology even though it was widely used in other disciplines a decade or two earlier.
Preprint servers also facilitate communication between disciplines, bridging
cultural as well as geographic divides. Examples include a recent series of
papers on the theory of natural selection that was posted to arXiv
simultaneously with its publication in the \emph{Journal of Evolutionary
Biology} \cite{JEB:JEB2431,JEB:JEB2498,JEB:JEB2378,JEB:JEB2373}. Other
submissions in this category include evolutionary and ecological theory by
authors trained in physics and computer science.  Since authors in these fields
regularly check arXiv, submitting preprints may be the most effective way
biologists can help others avoid ``repeated work'' \cite{de2011contribution} and
form a synthetic community of evolutionary theorists from disparate backgrounds.
The advantages are clear and the costs are low.

\section{Funding}

PDP is supported by an Alexander Graham Bell scholarship from the National
Sciences and Engineering Council of Canada. EPW is supported by a CAREER Award
from the National Science Foundation (DEB-0953694). JJA is supported by NSF
DEB-0614166 and NSF DEB-0919018. KR is supported by NSF DEB-1021553. DG is
funded by a Discovery Grand from the National Sciences and Engineering Council
of Canada and by the Canada Research Chair program.

\section{Acknowledgements}

We thank Carl Boettiger for helpful comments on an earlier version of this
manuscript.

\newpage
\bibliography{refs}

\end{document}

