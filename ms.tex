\documentclass[letterpaper,twocolumn,superscriptaddress,showkeys]{revtex4}
\usepackage[utf8]{inputenc}
\usepackage{color,dcolumn,graphicx,hyperref}
\hypersetup
{
    colorlinks = true, linkcolor = blue, citecolor = blue, urlcolor = blue,
}

\begin{document}

\title{Developing a preprint culture in biology}

\author{Philippe Desjardins-Proulx}
\email[E-mail: ]{philippe.d.proulx@gmail.com}
\affiliation{Theoretical Ecosystem Ecology laboratory, Universit\'e du Qu\'ebec \`a Rimouski, Canada.}
\affiliation{Quebec Center for Biodiversity Science, McGill University, Canada.}
\affiliation{D\'epartement des sciences biologiques, Universit\'e du Qu\'ebec \`a Montr\'eal, Canada.}

\author{Ethan P. White}
\affiliation{Departement of Bology, Utah State University, United-States of America.}

\author{Joel J. Adamson}
\affiliation{Ecology, Evolution and Organismic Biology, University of North
Carolina at Chapel Hill, United-States of America}

\author{Timoth\'ee Poisot}
\affiliation{Theoretical Ecosystem Ecology laboratory, Universit\'e du Qu\'ebec \`a Rimouski, Canada.}
\affiliation{Quebec Center for Biodiversity Science, McGill University, Canada.}
\affiliation{International Network for Next-Generation Ecology.}

\author{Karthik Ram}
\affiliation{Environmental Science, Policy, and Management. University of California, Berkeley. Berkeley, CA. United-States of America.}

\author{Dominique Gravel}
\affiliation{Theoretical Ecosystem Ecology laboratory, Universit\'e du Qu\'ebec \`a Rimouski, Canada.}
\affiliation{Quebec Center for Biodiversity Science, McGill University, Canada.}

\keywords{Publishing; preprint server; Green Open Access; arXiv.}

% No abstract needed.

\maketitle

\section{Introduction}

Public preprint servers allow authors to make manuscripts publicly
available before, or in parallel to, submitting them to journals for
traditional peer-review. The rationale for preprint servers is
fundamentally simple: to make the results available to the scientific
community as soon as possible rather than to wait until the
peer-review process is fully completed. The goal of arXiv and open
preprint servers is not to avoid the peer-review process; almost all
manuscripts on arXiv are submitted to peer-review.  Sharing ideas as
quickly as possible using preprint servers has numerous advantages.
These include rapid dissemination of work in progress to a wider
audience, including readers in developing countries where access to
subscription journals is often a limiting factor in scientific
research.  Furthermore, preprints provide the opportunity to solicit
feedback from a larger pool of reviewers and can be seen as an
integral aspect of a vigorous peer-review process \cite{hoc12}.
Preprint servers therefore increase the number of opportunities for
review and revision prior to publication, resulting in higher quality
submissions and also could alleviate reviewer burden.

Preprints gained popularity 20 years ago with the advent of arXiv, an
open preprint server widely used in physics and mathematics
\cite{gin11}. Preprints are also integral to the culture of other
scientific fields.  Paul Krugman noted that, in economics, the
\emph{traditional model of submit, get refereed, publish, and then
  people will read your work broke down a long time ago. In fact, it
  had more or less fallen apart by the early 80s} \cite{kru12}. In
addition to a section in arXiv, economists have also the RePEc
(Research Papers in Economics) initiative, which aims to create an
archive of working papers, manuscripts, and book chapters.  Despite
the success of this approach in other fields, most manuscripts in
biology are never submitted to any public archive.
% JJA: This is a tenuous statement: MANY if not most biomedical labs
% in the US are funded by NIH, which requires submission to Pubmed
% Central, which many would consider a preprint archive.  In other
% words, a biomedical editor might take issue with this statement,
% especially if he considers "biology" to be just his part of biology
% --- I know plenty of people who only think of "biology" as
% biomedical science
In this article, we will first highlight the advantages of open
preprint servers for both scientists and publishers. We will also
debunk a few misconceptions, discuss the policies of major publishers
in biology, and briefly review the most popular open preprint servers.

\section{The case for open preprints}

The first and most often discussed advantage of arXiv and open
preprints is speed (Figure~\ref{fig:map}). The time between submission
and the official publication of a manuscript can be measured in
months, sometime in years. For all this time, the research is known
only to a select few: colleagues, editors, reviewers.  Thus, the
science cannot be used, discussed, or reviewed by the wider scientific
community.  This stage of manuscript preparation is a problem for both
scientists and publishers.  Manuscripts that are unknown cannot be
used and thus take more time to be cited. It has been shown that
high-energy physics, with its high ArXiv submission rate, had the
highest immediacy among physics and mathematics \cite{pra05}.
% JJA: Can you say what "immediacy" means?

Furthermore, the review process as a whole is critically over-loaded,
because as the number of active scientists increases, because the
pressure to publish increases, and because of an effect dubbed ``the
tragedy of the reviewers commons'' REF.  At the same time, rejection
rates are high in most journals (REF), and when not invited to submit
a revision, authors must start the whole process all over again.
Initiatives to reduce time from submission to publication have emerged
across the scientific community XXX et coll.  (REF) called for the
recycling and reuse of peer-reviews: by attaching previous reviews and
detailed replies to a new submission, both the editor and the referees
can gauge the work done on the manuscript, and perhaps evaluate it
with less prejudice. In a similar way, the \emph{Peerage of Science}
initiative allows authors to seek anonymous pre-review by their
peers. Some journals now accept to publish papers which received good
evaluations, with \emph{Animal Biology} having recently accepted first
a paper reviewed entirely with the \emph{Peerage of Science}
\cite{abb12}, effectively outsourcing the review process. A widespread
use of preprint servers can achieve the same goal of reducing the time
spent in review. By putting a manuscript for open comments and
criticisms, the authors will receive valuable feedback and can improve
the version which will be submitted. With a rich enough community of
scientists depositing preprints, and commenting on them, the process
of an open pre-review can become widespread and will overall increase
the quality of first submissions.

\begin{figure}[ht!] \centering\includegraphics[width=0.50\textwidth]{map.pdf}
\caption { It can take several months, and even a few years, before a submitted
paper is officially published and citable.  The average time to publication
varies greatly between journals and can be as low as 104 days (Evolution for
2011) to 213 (PLOS One in 2010).  Meanwhile, few people are aware of the
research that has been done since, typically, only close colleagues are given
access to the preprints. With public preprint servers, the science is
immediately available and can be openly discussed, analyzed, and integrated into
current research. It benefits both science and publishers. Both want the papers
to be well-known and cited, and public preprints make it possible to integrate
research even before publication, greatly improving immediacy.  }
\label{fig:map} \end{figure}

Public preprint servers offer a much fairer way to establish
intellectual priority by making the work available when done. Some
manuscripts will spend much more time in the review process than
others. Surprisingly, there is a perception in biology that public
preprints make it easier to steal ideas, as if a scientific idea was
only complete once published in a peer-reviewed journal.
Mathematicians and physicists have embraced arXiv in part to establish
priority in a fair way \cite{cal12}.
% JJA: I snipped the part about getting "snooped" (do you mean
% "scooped"?) as it seems normative and might annoy some readers.  The
% "should" should be implicit.

Prepublication reviews by a small network of colleagues is an
important part of the scientific process, attested by the fact that
nearly all published papers acknowledge comments by people not listed
as co-authors.  Preprint servers simply offer a way to extend this
network of colleagues to the entire scientific community. It ensures
that science is not constrained by small networks of scientists
exchanging ideas.  Paul Ginsparg created arXiv.org in part for
democratic reasons: he wanted everyone from students in small
universities to Ivy-League professors to have access to the most
recent scientific \emph{ideas}.  Ginsparg's revolutionary idea was
simply to use the power of the internet for preprints, not just for
the end product, so the scientific process can be open as soon as
possible.

\section{Preprints in biological sciences}

Submitting to a preprint server is becoming more common in biology,
even though it still involves a minority of papers. The quantitative
biology section in arXiv is experiencing faster growth in submissions
than any other field \cite{cal12}.  Examples include a recent series
of papers on the theory of natural selection that was posted to ArXiv
simultaneously with its publication in the \emph{Journal of
  Evolutionary Biology}
\cite{JEB:JEB2431,JEB:JEB2498,JEB:JEB2378,JEB:JEB2373}.  Other
submissions in this category include evolutionary and ecological
theory by authors trained in physics and computer science.  Since
authors in these fields regularly check ArXiv, submitting preprints
may be the most effective way biologists can help others avoid
``repeated work'' \cite{de2011contribution} and form a synthetic
community of evolutionary theorists from disparate backgrounds.

Most scientific journals are preprint-friendly: Nature, PLOS, BMC,
PNAS, Science (mostly) \ref{table:policies}, and all the journals from
Elsevier and Springer.  The Ecological Society of America and the
Genetics Society of America recently changed their policies to allow
public preprints.  Few scientific publications will not consider a
manuscript submitted to a public preprint server.  Still, a few
journals adopt a ``by default'' hostile attitude towards preprints,
mostly due to the lack of clear policy of the publishers, or perhaps
because a preprint culture has not developed in biology and the
practice is still considered unusual. As an example, Wiley-Blackwell,
which publishes some leading journals, has no official policy on the
subject \ref{table:policies}.

Part of the hostility to preprints servers comes from a certain
interpretation of the Ingelfinger rule: scientists should not publish
the same manuscript twice \cite{alt96}. However, papers submitted to
arXiv or other preprints servers are only published in one
peer-reviewed journal. A preprint is just a preprint, it is a document
that allows ideas to spread and be discussed, but it is not yet
formally validated by the peer-review system, which is why the
majority of publishers does not see arXiv and similar services as a
violation of the Ingelfinger rule. \emph{Nature} responded to the
rumour that they refused manuscripts submitted to arXiv by saying that
``\emph{Nature} never wishes to stand in the way of communication
between researchers. We seek rather to add value for authors and the
community at large in our peer review, selection and editing''
\cite{nat05}.

% Discuss embargoes?
%A second potential problem is embargoes. Many publishers subject their article
%to embargoes \cite{rea12}.

\begin{table*}
    \centering
    \begin{tabular}{|ll|}
    \hline
    Publisher                                   & Policy \\
    \hline
    Springer                            	& Accept \\
    BMC                                 	& Accept \\
    Elsevier                            	& Accept \\
    Nature Publishing Group             	& Accept \\
    Public Library of Science           	& Accept \\
    Genetics Society of America                 & Accept \\
    Royal Society                       	& Accept \\
    National Academy of Science (USA)           & Accept \\
    Ecological Society of America       	& Accept \\
    Oxford Journals                             & Accept \\
    Science                             	& Ambiguous \\
    Wiley-Blackwell                       	& No general policy \\
    British Ecological Society                  & No answer to our query \\
    \hline
    \end{tabular}
    \caption{Policies for important publishers in biology. Some publishers
tolerate preprints except for a few of their medical journals, e.g.: Journal
of the National Cancer Institute from Oxford and The Lancer from Elsevier.}
    \label{table:policies}
\end{table*}

% Not sure about this paragraph, as much as I love to hate MS Office, I would
% rather keep the thing as a "pro-preprint" paper only and not use it to promote
% LaTeX/R.
% -- PhDP

% JJA: It's not crucial.  My main point was that there are certain
% non-obvious cultural reasons that arXiv submissions are common.
% It's not just because arXiv started in theoretical physics.  It's
% also that people who use TeX tend to have a certain attitude, I
% would argue inherited from Don Knuth, and people who use TeX and
% LaTeX _talk to each other_.  So perhaps this was a polite way of
% saying that people in math and physics are more community-oriented,
% and people in biology tend to walk around paranoid about getting
% scooped.  Some people also might think that they can't submit to
% arXiv because they use Word.

%There are other barriers to adoption of preprint servers by biologists.  The
%most notable preprint server, arXiv, is designed to work best with \LaTeX, the
%dominant document preparation system among physicists and mathematicians.
%Jackson \cite{jackson2002preprints} argues that \LaTeX introduced an
%open source mindset to its users, who now freely share their research
%findings as well as their software.  Many biologists instead use proprietary
%software to prepare their research findings, and many journals officially prefer
%Microsoft Word documents as submissions.  The recent discipline-wide adoption of
%free software packages such as R and associated document generation systems
%\cite{xie2012}, as well as interest in open access publishing, has coincided
%with recent rise in use of preprint servers by biologists, supporting Jackson's
%claims \cite{xie12}.

\section{Current offerings}

We briefly discuss the main options to submit preprints to open servers:
arXiv.org, Figshare, and the upcoming PeerJ and F1000Research.

\subsection{arXiv}

arXiv (\url{http://arxiv.org/}) is the most widely-used preprint server today,
and its use is almost universal in some branches of mathematics and physics.
% Ni!!!!  I really don't like starting sentences with "it"
arXiv provides a reliable citation system for all eprints and is
especially popular in high-energy physics. Physicist Paul Ginsparg
created arXiv in 1991 for theoretical high-energy physicists to
communicate preprints via email and ftp, and soon thereafter adopted
the newly created world-wide web\cite{jackson2002preprints}.  ArXiv
now receives over 7,000 submissions per month
(\url{http://arxiv.org/show_monthly_submissions}) and divides its
submissions into subcategories of physics, mathematics, computer
science, quantitative biology, finance and statistics.  The
quantitative biology category includes subcategories for Populations
and Evolution, Quantitative Methods and other categories that may be
of interest to biologists.

Submission to arXiv is fully automated.  Authors can submit
\TeX{}/\LaTeX{} documents that are compiled on the server or directly
submit in PDF/PS format (for example, as exported by a word
processor).  A moderation system was put in place in 2004: papers must
be categorized by an endorser. At least one author of a paper must be
an endorser that has previously submitted a paper or has received
permission to submit to a particular category.  Many authors in
mathematics and physics submit papers as soon as they are ready for
review by colleagues, although another popular option is submitting
simultaneously to a journal and arXiv.

% To review: 
Authors must either have their arXiv submission available under an
open license or grant arXiv a non-exclusive and irrevocable license to
distribute the work. In either case, arXiv does not require copyright
transfer and only requires the rights to distribute submitted articles
in perpetuity. Thus, submitting to arXiv does not in itself prevent
the authors from transferring their rights to a publisher, which is
why most publishers tolerate arXiv, even though they ask the authors
to transfer their right to them upon acceptance of the article.

Most papers posted to arXiv are eventually printed in journals but
there are notable exceptions, such as Perelman's landmark paper
leading to the proof of the Poincar\'{e} conjecture
\cite{2002math.....11159P}.  However, arXiv has never sought to
replace scientific journals and explicitly states that it serves a
different function as ``an openly accessible, moderated repository for
scholarly articles in specific scientific disciplines.'' arXiv is now
administered by the Cornell University Libraries, with funding coming
from voluntary pledges by academic institutions along with matching
funds from the Simons Foundation \cite{arxiv_future}.  One-hundred
twenty six of the top two-hundred institutions in terms of downloads
have provided the operating budget for arXiv over the next five years.
This plan reduces the financial burden on Cornell University and
transfers governance to a collaborative community in accordance with
arXiv's key principles.  arXiv takes numerous measures to ensure that
the repository will remain permanently available and submissions will
be readable.

\subsection{Figshare}

Figshare (\href{http://figshare.com/}{http://figshare.com/}) is an
open server that allows scientists to submit any research output:
manuscript, figures, datasets, videos, theses, presentations, and so
on. There are no rules to limit what constitutes a research output
and, unlike arXiv, there is no endorser system. All figshare content
has a unique digital object identifier (DOI) like any journal article,
thus offering a permanent and stable link to the content.  A flexible
tag system is used to classify each item. All content can be commented
and is licensed under the Creative Commons (CC-BY) license, except
datasets which are published under CC0. The CC-BY license grants
rights to \emph{copy, distribute, display and perform the work and
  make derivative works based on it only if they give the author or
  licensor the credits in the manner specified by these.}  CC0 is the
most permissive license and effectively puts the work in public domain
(no rights reserved) or, if it is not possible in the given
jurisdiction, provides a simple permissive license.

One of the biggest advantage of figshare over arXiv is that is it not
limited to quantitative sciences. arXiv.org has sections on
quantitative biology but might not be appropriate for non-quantitative
work. With its flexible approach to preprints, figshare offers an
important alternative to arXiv for empirical biologists. Furthermore,
by allowing all types of content, figshare arguably provides an
archive for early results (e.g.: figures, lab presentations).

\subsection{PeerJ}

PeerJ (\href{https://peerj.com/}{https://peerj.com/}) is a new
publishing system that provides both a preprint server, and a peer
reviewed journal.  It is focused on the the biological and medical
sciences, which may help overcome the perception that preprints do not
have a home in biology.  PeerJ allows commenting on posted preprints,
improving the potential for pre-publication dialog. In addition,
preprints can be made private if the authors choose, and shared only
with selected colleagues. While this reduces some of the benefits of
preprints described above, it may allow some researchers who would not
otherwise post preprints to begin to explore the possibility in a
manner appropriate to their current circumstances.

In contrast to other preprint servers users cannot post unlimited
public preprints for free. One preprint per year can be posted for
free and a onetime (i.e. lifetime) fee of 99 dollars allows the
posting of unlimited public preprints. It is also worth noting that
the preprint server is not tied to the journal, so preprints can be
posted regardless of where they will eventually be submitted for
publication.

\subsection{F1000Research}

F1000Research is not a public preprint server like the previous three
servers.  Whereas arXiv, Figshare, and PeerJ offer an option to submit
a manuscript without having it reviewed, papers submitted to
F1000Research will eventually be reviewed. Thus, F1000Research offers
a hybrid model with publicly available manuscripts at time of
submission and standard peer-reviews. Manuscripts are considered
``accepted'' and will only be indexed after two positive referee
response.

\subsection{Github et al.?}

This manuscript was developed entirely as an open project on
github. github is one of several hosting services for collaborative
development using the git revision control software.  git is a
decentralized revision control system created by Linus Torvalds used
primarily to develop software, including the Linux kernel. It allows
the users to share code, track changes, and merge versions. For
example, during the development of this manuscript, one author would
fork the project (that is: make a personal copy), modify it, and then
the changes were merged back into the main branch. These techniques
are widely used in open source software development. It goes even
farther than submitting preprints by opening the entire writing
process.

\section{Conclusion}

Open preprint servers offer a great opportunity for open science,
especially if the community embraces the idea of discussing
preprints. Initiatives like Haldane's Sieve
(\href{http://haldanessieve.org/}{http://haldanessieve.org/}), a new
blog discussing arXiv papers in population genetics, will help make
arXiv attractive for scientists looking to promote their work. These
initiatives are important to fully exploit the potential of open
preprints servers. Posting preprints online increases the community of
available informal peer reviewers, and uses the internet for its
original community-building purposes.  Preprint servers also
facilitate communication between disciplines, bridging cultural as
well as geographic divides. The advantages are clear and the costs are
low.

\section{Funding}

PDP is funded by an Alexander Graham Bell scholarship from the National Sciences
and Engineering Council of Canada.

DG is funded by a Discovery Grand from the National Sciences and Engineering
Council of Canada and by the Canada Research Chair program.

JJA is supported by NSF DEB-0614166 and NSF DEB-0919018.

\section{Acknowledgements}

...

\newpage
\bibliography{refs}
\bibliographystyle{plain}

\end{document}

