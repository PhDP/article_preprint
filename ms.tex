% Template for PLoS
% Version 1.0 January 2009
%
% To compile to pdf, run:
% latex ms
% bibtex ms
% latex ms
% latex ms
% dvipdf ms

\documentclass[10pt]{article}

% amsmath package, useful for mathematical formulas
\usepackage{amsmath}
% amssymb package, useful for mathematical symbols
\usepackage{amssymb}

% graphicx package, useful for including eps and pdf graphics
% include graphics with the command \includegraphics
\usepackage{graphicx}

% cite package, to clean up citations in the main text. Do not remove.
\usepackage{cite}

\usepackage{color} 

% Use doublespacing - comment out for single spacing
\usepackage{setspace}
\usepackage{hyperref} 
\doublespacing

% Text layout
\topmargin 0.0cm
\oddsidemargin 0.5cm
\evensidemargin 0.5cm
\textwidth 16cm 
\textheight 21cm

% Bold the 'Figure #' in the caption and separate it with a period
% Captions will be left justified
\usepackage[labelfont=bf,labelsep=period,justification=raggedright]{caption}

% Use the PLoS provided bibtex style
\bibliographystyle{plos2009}

% Remove brackets from numbering in List of References
\makeatletter
%\renewcommand{\@biblabel}[1]{\quad#1.}
\makeatother

% Leave date blank
\date{}

\pagestyle{myheadings}
%% ** EDIT HERE **


%% ** EDIT HERE **
%% PLEASE INCLUDE ALL MACROS BELOW

%% END MACROS SECTION

\begin{document}

% Title must be 150 characters or less
\begin{flushleft}
{\Large
\textbf{The Case for Open Preprints in Biology}
}
% Insert Author names, affiliations and corresponding author email.
\\
Philippe Desjardins-Proulx$^{1, 2,\ast}$, 
Ethan P. White$^{3}$, 
Joel J. Adamson$^{4}$,
Karthik Ram$^{5}$,
Timoth\'ee Poisot$^{1,2,6}$,
Dominique Gravel$^{1,2}$
\\
\bf{1} Theoretical Ecosystem Ecology laboratory, Universit\'e du Qu\'ebec \`a Rimouski, Quebec, Canada.
\\
\bf{2} Quebec Center for Biodiversity Science, McGill University, Quebec, Canada.
\\
\bf{3} Departement of Biology, Utah State University, Utah, United-States of America.
\\
\bf{4} Ecology, Evolution and Organismic Biology, University of North Carolina at Chapel Hill, North Carolina, United-States of America.
\\
\bf{5} Environmental Science, Policy, and Management, University of California, Berkeley, California, United-States of America.
\\
\bf{6} International Network for Next-Generation Ecology.
\\
$\ast$ E-mail: philippe.d.proulx@gmail.com
\end{flushleft}

\section*{Introduction}

Public preprint servers allow authors to make manuscripts publicly available
before, or in parallel to, submitting them to journals for traditional peer-
review. The rationale for preprint servers is fundamentally simple: to make
the results of research available to the scientific community as
soon as possible, instead of waiting until the peer-review process is fully
completed.  Sharing manuscripts using preprint servers has numerous advantages
including: 1) rapid dissemination of work-in-progress to a wider audience; 2)
immediate visibility of the research output for early-career scientists; 3)
improved peer review by encouraging feedback from the entire research
community; and 4) a fair and straightforward way to establish precedence.

Open preprint servers offer a great opportunity for open science, especially if
the community embraces the idea of discussing preprints. Initiatives like
Haldane's Sieve (\href{http://haldanessieve.org/}{http://haldanessieve.org/}), a
new blog discussing arXiv papers in population genetics, can help make arXiv
attractive for scientists looking to promote their work \cite{lom12}. These
initiatives are important to fully exploit the potential of open preprint
servers. Posting preprints online increases the community of available informal
peer reviewers, and uses the internet for its original community-building
purposes.

Preprints began to gain popularity 20 years ago with the advent of arXiv, an
open preprint server widely used in physics and mathematics \cite{gin11}.
Preprints are also integral to the culture of other scientific fields.  Paul
Krugman noted that, in economics, the \emph{traditional model of submit, get
refereed, publish, and then people will read your work broke down a long time
ago. In fact, it had more or less fallen apart by the early 80s} \cite{kru12}.
In addition to a section on arXiv, economists have the RePEc (Research Papers in
Economics) initiative, which aims to create an archive of working papers,
manuscripts, and book chapters.

Despite the success of this approach in other fields, most manuscripts in
biology are not posted to preprint servers and are therefore not seen by more
than a handful of other scientists prior to publication. In this article, we
highlight the advantages of open preprint servers for both scientists and
publishers, discuss the preprint policies of major publishers in biology, and
describe the main options to publish preprints (Box 1, Table 1).

\section*{The case for public preprints}

The first and most often discussed advantage of open preprints is
speed (Figure 1). The time between submission and the official
publication of a manuscript can be measured in months, sometimes in years. For
all this time, the research is known only to a select few: colleagues, editors,
reviewers. Thus, the science cannot be used, discussed, or reviewed by the wider
scientific community. In a recent blog post, C. Titus Brown noted how posting a
paper on arXiv quickly led to a citation (arXiv papers can be cited) and his
research was used by another researcher \cite{bro12}. The current system of
hiding manuscripts before acceptance pose problems for both scientists and
publishers. Manuscripts that are unknown cannot be used and thus take more time
to be cited. It has been shown that high-energy physics, with its high arXiv
submission rate, had the highest immediacy among physics and mathematics
\cite{pra05}. Immediacy measures how quickly articles are cited. 

Public preprints can be crucial to early-career scientists. The delay before
publication is seldom compatible with the pressure to show an impressive
publication record when applying for a scholarship or a position. Increasing the
perceived value of pre-prints as close, or equal, to journal articles will allow
young researchers to put their research outcome in the open, and build a
reputation for themselves through the diffusion of their work without fear that
this work will not be recognized by grant or job committees.

Posting manuscripts as preprints also has the potential to improve the 
quality of science by allowing prepublication feedback from a large pool of 
reviewers. In our experience, prepublication reviews by a small network of 
colleagues are common in the biological sciences and form an important part 
of the scientific process. These ``friendly'' reviews increase the chance 
of errors being caught prior to publication. Furthermore, the formal 
peer-review process as a whole is critically over-loaded. As the number of 
active scientists increases and the pressure to publish increases, it is 
becoming difficult for journals to find reviewers \cite{hoc09}.  At the same 
time, rejection rates are high in most journals \cite{aar08,roh09}, and when 
not invited to submit a revision, authors must start the process over again 
at another journal. As a result, initiatives to reduce time from submission 
to publication have emerged across the scientific community. Rohr et al.  
\cite{roh09} called for the recycling and reuse of peer-reviews: by 
attaching previous reviews and detailed replies to a new submission, both 
the editor and the referees can gauge the work done on the manuscript, and 
perhaps evaluate it with less prejudice. A widespread use of preprint 
servers can achieve the same goal of reducing the time spent in review. With 
a rich enough community of scientists depositing preprints, and commenting 
on them, the process of an open pre-review can become widespread and will 
overall increase the quality of first submissions \cite{hoc12}.

Finally, public preprint servers offer a fair way to establish intellectual
priority by making the work available as soon as it is complete. Some
manuscripts will spend much more time than others in the review process and/or in
production after acceptance. This means that publication and
acceptance dates do not accurately characterize who came up with an idea
first. For this reason, mathematicians and physicists have embraced arXiv in
part to establish priority in a fair way \cite{gin11,cal12}.

\section*{Preprints in biological sciences}

In contrast to other disciplines, the field of biology has effectively no
preprint culture, with the exception of small pockets of primarily highly
quantitative research (\emph{e.g.}, epidemiology, population genetics).  While
submitting to preprint servers has become more common in the past few years, the
number of biology papers submitted to preprint servers still represents only a
small fraction of the total research produced in biology (Figure 2).

There are a number of reasons why biologists have not developed a culture of
sharing preprints, many of which are based on common misconceptions. For
example, in contrast to other fields there is a perception in biology that
public preprints make it easier to steal ideas \cite{gin11}. In other fields
preprints serve the opposite role, they allow straightforward establishment
of precedence, letting research lay claim to an idea thus preventing it from
being ``stolen'' \cite{gin11}. Another major concern is based on a certain
interpretation of the Ingelfinger rule: scientists should not publish the
same manuscript twice \cite{alt96}. A preprint is simply a document that
allows ideas to spread and be discussed, it is not yet formally validated by
the peer-review system. This is why almost all the major publishers in
biology are preprint-friendly, including: Nature Publishing Group, PLOS,
BMC, PNAS, Elsevier, and Springer (Table 2). This year, both the Ecological
Society of America and the Genetics Society of America changed their
policies to allow public preprints. \emph{Nature} even felt compelled to
respond to the rumour that they refused manuscripts submitted to arXiv by
saying that ``\emph{Nature} never wishes to stand in the way of
communication between researchers. We seek rather to add value for authors
and the community at large in our peer review, selection and editing'' \cite
{nat05}. Still, a few journals adopt a ``by default'' hostile attitude
towards preprints, mostly due to the lack of clear policy of the publishers.
As an example, Wiley-Blackwell, which publishes some of the leading journals
in biology, has no official policy on the matter.

\section*{Box 1: Preprint Server Roundup}

\subsection*{arXiv}

arXiv (\url{http://arxiv.org/}) is the most widely-used preprint server today,
and its use is almost universal in some branches of mathematics and physics.
arXiv has a system of moderators and endorsers. At least one author of a paper
must be an endorser that has either previously submitted a paper or has received
permission to submit.  Moderators have the power to change the classification of
a manuscript.

\subsection*{figshare}

figshare (\href{http://figshare.com}{http://figshare.com}) is an open server
allowing scientists to submit any research output: manuscript, figures,
datasets, videos, theses, presentations, and so on. There are no rules to limit
what constitutes a research output and, unlike arXiv, there is no endorser
system. A flexible tag system is used to classify each item.

\subsection*{PeerJ}

PeerJ (\href{https://peerj.com/}{https://peerj.com/}) is a new commercial open
access publisher focused on the biological sciences that provides a preprint
server and a peer reviewed journal. Preprints can optionally be made private.
One preprint per year can be posted for free, with a onetime (\emph{i.e.}
lifetime) fee for unlimited public preprints. Preprints can be posted to PeerJ
regardless of where they will be submitted for publication.

\subsection*{F1000Research}\label{f1000r}

Whereas arXiv, figshare, and PeerJ offer an option to submit a manuscript
without having it reviewed, papers submitted to F1000Research will eventually be
reviewed. Thus, F1000Research offers a hybrid model with publicly available
manuscripts at time of submission and standard peer-reviews that occur as part
of the submission process. Manuscripts are considered ``accepted'' and will only
be indexed after two positive referee responses.

\subsection*{GitHub}

This manuscript was developed entirely as an open project on GitHub. GitHub is
one of several hosting services for collaborative development using the Git
version control system (VCS). It allows numerous contributers to work
asynchronously on the same project, often in parallel branches, all of which can
be effortlessly merged and version controlled.  Git is primarily used for
software development \cite{aru12} but it provides a powerful tool way to
collaborate on every step of the manuscript development process \cite{ram13}.

\subsection*{Other options}

Scientific publishing is more diversified than ever. There are now many 
alternative options to submit articles before formal publications. For 
example, social networks such as ResearchGate can be used to submit 
preprints \cite{lin12}. Also, if GitHub pushes openness further by opening 
the writing process, open notebooks go even further by opening the entire 
scientific process \cite{san11}. 

\section*{Conclusion}

The ongoing discussions on the publication process, peer-reviewing and 
alternative publication models are all symptoms of the current uneasiness 
with the ever growing obsession with bibliographic metrics such as the 
impact factor \cite{Fisher2012}. Researchers are pressured to orient their 
publication strategy to maximize their number of publications and total 
citations. A well-known consequence is to submit manuscripts first to the 
most prestigious journals, and then resubmit to ``lower level'' journals as 
they are rejected. The numerous negative impacts of such behavior have been 
discussed in depth \cite{hoc09} and include a long delay between the time a 
manuscript is finished to its publication. Research activities and the 
publication process are drifting away from their fundamental object, namely 
the diffusion of novel scientific discoveries. 

Developing a preprint culture in biology will not solve all problems with 
the current publication process. However, it might significantly reduce its 
negative consequences. The role of peer-reviewing is to judge the scientific 
quality of a study. It is the first barrier against the fraudulent and poor 
quality science susceptible to impede scientific progress. In practice, the 
peer-review system is not only used to evaluate scientific quality but also 
to judge pertinence. On the other hand, preprints are not filtered, neither 
for their quality nor their pertinence. Widespread adoption of preprint 
servers has the potential to shift the diffusion strategy: journals would 
remain important to validate publications, but the relevance of a study 
should only be judged by many more readers than the typical two-four 
anonymous reviewers. With a shift in the diffusion strategy, the role of 
traditional journals and their editors would be to showcase scientific 
discoveries for specialized readership. 

Making publication easier can lead to the proliferation of studies of uneven 
quality. A trade-off between the intensity of the peer-review filtering and 
the benefits to science has been hypothesized \cite{Aarssen2012}.  With 
increasingly stringent peer reviewing, the quality of published papers can 
improve at the cost of an increased load on authors and 
reviewers and greater delays for publication.  Preprints are simply 
bypassing this model for what we believe is the progress of science: they 
speed up the dissemination of scientific discoveries and put on reader's 
shoulders the responsibility to judge originality and pertinence.

\section*{Acknowledgements}

We dedicate this article to Aaron H. Swartz (1986-2013).  We thank Carl
Boettiger, Mark Hahnel, and Hedvig Nenz\'en for helpful comments on an earlier
version of this manuscript.

\section*{References}
\bibliography{refs}

\newpage
\section*{Figure Legends}

\subsection*{Figure 1 (map.tif)}

It can take several months before a submitted paper is officially published and
citable. Meanwhile, few people are aware of the research that has been done
since, typically, only close colleagues are given access to the preprints. With
public preprint servers, the science is immediately available and can be openly
discussed, analyzed, and integrated into current research.

\subsection*{Figure 2 (arxiv.tif)}

Submissions to the quantitative biology section lag behind
physics, mathematics, and computer science.

\newpage
\section*{Tables}

\begin{table}[!ht]
  \caption{\bf{Popular options for preprints}}
  \begin{tabular}{|l|ccccccc|}
  \hline
  Website               & Free & Comments & Private & Peer-reviewed & DOI & Version-Control & Other Content\\
  \hline
  arXiv.org             & Yes & No & No & No & No & No & No\\
  figshare.com          & Yes & Yes & Yes & No & Yes & No & Yes\\
  peerj.com             & 1/yr & Yes & Yes & No & Yes & No & No\\
  f1000research.com     & No & Yes & No & Yes & Yes & No & No\\
  github.com            & Yes & Yes & Yes & No & No & Yes & Yes\\
  \hline
  \end{tabular}
  \label{table:box1}
\end{table}

\begin{table}[!ht]
  \caption{\bf{Policies for important publishers in biology.}}
  \begin{tabular}{|ll|}
  \hline
  Publisher                             & Policy \\
  \hline
  Springer                              & Accept \\
  BMC                                   & Accept \\
  Elsevier                              & Accept \\
  Nature Publishing Group               & Accept \\
  Public Library of Science             & Accept \\
  Genetics Society of America           & Accept \\
  Royal Society                         & Accept \\
  National Academy of Science (USA)     & Accept \\
  Ecological Society of America         & Accept \\
  Oxford Journals                       & Accept \\
  Science                               & Ambiguous \\
  Wiley-Blackwell                       & No general policy \\
  British Ecological Society            & No answer to our query \\
  \hline
  \end{tabular}
  \label{table:policies}
\end{table}

\newpage
\section*{Table Legends}

\subsection*{Table 1}

\textbf{Free:} Can preprints be submitted for free.
\textbf{Comments:} Support for online comments.
\textbf{Private:} Support for private preprints.
\textbf{Peer-reviewed:} Whether the preprints are peer-reviewed on the server.
\textbf{DOI:} Each item is assigned a unique digital object identifier.
\textbf{Version-Control:} Is the preprint stored using a version-control system with the complete history of modifications?
\textbf{Other content:} Can upload figures, videos, datasets, code.

\subsection*{Table 2}

Some publishers tolerate preprints except for a few of their medical
journals, for example the \emph{ Journal of the National Cancer Institute}
from Oxford and \emph{The Lancet} from Elsevier. Data from \cite{war12}.

\end{document}
