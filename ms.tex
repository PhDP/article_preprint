\documentclass[letterpaper,twocolumn,superscriptaddress,showkeys]{revtex4}
\usepackage[utf8]{inputenc}
\usepackage{color,dcolumn,graphicx,hyperref}
\hypersetup
{
    colorlinks = true, linkcolor = blue, citecolor = blue, urlcolor = blue,
}

\begin{document}

\title{Open Preprints in Ecology \& Evolution}

\author{Philippe Desjardins-Proulx}
\email[E-mail: ]{philippe.d.proulx@gmail.com}
\affiliation{Theoretical Ecosystem Ecology laboratory, Universit\'e du Qu\'ebec \`a Rimouski, Canada.}
\affiliation{Quebec Center for Biodiversity Science, McGill University, Canada.}
\affiliation{D\'epartement des sciences biologiques, Universit\'e du Qu\'ebec \`a Montr\'eal, Canada.}

\author{Ethan P. White}
\affiliation{Departement of Bology, Utah State University, United-States of America.}

\author{Joel J. Adamson}
\affiliation{Ecology, Evolution and Organismic Biology, University of North
Carolina at Chapel Hill, United-States of America}

\author{Timoth\'ee Poisot}
\affiliation{Theoretical Ecosystem Ecology laboratory, Universit\'e du Qu\'ebec \`a Rimouski, Canada.}
\affiliation{Quebec Center for Biodiversity Science, McGill University, Canada.}
\affiliation{International Network for Next-Generation Ecology.}

\author{Karthik Ram}
\affiliation{Environmental Science, Policy, and Management. University of California, Berkeley. Berkeley, CA. United-States of America.}

\author{Dominique Gravel}
\affiliation{Theoretical Ecosystem Ecology laboratory, Universit\'e du Qu\'ebec \`a Rimouski, Canada.}
\affiliation{Quebec Center for Biodiversity Science, McGill University, Canada.}

\keywords{Publishing; preprint server; arXiv; Green Open Access.}

\begin{abstract}

...
 
\end{abstract}

\maketitle

\section{Introduction}

Public preprints servers allow authors to make their manuscripts publicly
available before, or in parallel to, submitting them to journals for traditional
peer-review. This idea gained popularity 20 years ago with the advent of arXiv,
an open preprint server widely used in the physical sciences \cite{gin11}, but
it also part of other fields' culture. Paul Krugman noted that, in economics,
the \emph{traditional model of submit, get refereed, publish, and then people
will read your work broke down a long time ago. In fact, it had more or less
fallen apart by the early 80s} \cite{kru12}. In addition to a section in arXiv,
economists have also the RePEc (Research Papers in Economics) initiative, which
aims to create an archive of working papers, manuscripts, and book chapters.

The idea behind preprint servers is fundamentally simple: to make the results
available to the scientific community as soon as possible rather than wait until
the peer-review process is fully completed. The point of arXiv and open
preprints servers is not to avoid the peer-review process. Almost all
manuscripts on arXiv are submitted to peer-review.  Sharing drafts of research
papers on preprint servers has numerous advantages. These include rapid
dissemination of work in progress to a wider audience that includes not only
scientists and the general public, but also readers in developing countries
where access to subscription journals are often a limiting factor. Further,
preprints provide the opportunity to solicit feedback from a larger pool of
reviewers. As a consequence, preprint servers increase the number of
opportunities for review and revision prior to publication, resulting in higher
quality submissions which could also alleviate reviewer burden.

Yet, despite the success of the approach in other fields of inquiry, most
manuscripts in ecology and evolution are never submitted to any public archive.
In this article, we will first highlight the advantages of open preprints
servers for both scientists and publishers. We will also debunk a few
misconceptions, discuss the policies of major publishers in ecology an
evolution, and briefly review the most popular open preprint servers.

\section{The case for open preprints}

The first and most often discussed advantage of arXiv and open preprints is
speed \ref{fig:map}. The time between submission and the official publication of
a manuscript can be measured in months, sometime in years. For all this time,
the research is only known to a select few: colleagues, editors, reviewers.
Thus, the science cannot be used, discussed, or reviewed by the wider scientific
community. It is a problem for both scientists and publishers. Manuscripts that
are unknown cannot be integrated and thus take more time to be cited.

The review process as a whole is critically over-loaded, because the number of
active scientists increases, because the pressure to publish increases, and
because of an effect dubbed ``the tragedy of the reviewers commons'' REF.  In
the same times, rejection rates are high in most journals (REF), and when the
not invited to submit a revisions, authors are left with the impression that
they must start the whole process all over again. It's thus no surprise that
different initiatives emerged over the last few years, to decrease the time
spent in review. XXX et coll. (REF) called for the recycling and reuse of
peer-reviews: by attaching previous reviews, and detailed replies, to a new
submission, both the editor and the referees can gauge the work done on the
manuscript, and perhaps evaluate it with less prejudice. In a similar way, the
\emph{Peerage of Science} initiative allows authors to seek anonymous pre-
review by their peers. Some journals (LIST?) now accept to publish papers
which received good evaluations, effectively outsourcing the review process. A
widespread use of preprint servers can achieve the same goal of reducing the
time spent in review. By putting a manuscript out there for open comments and
criticisms, the authors will receive valuable feedback, and can improve the
version which will be submitted. With a rich enough community of scientists
depositing preprints, and commenting on them, the process of an open pre-
review can become widespread, and will overall increase the quality of first
submissions.

\begin{figure}[ht!] \centering\includegraphics[width=0.50\textwidth]{map.pdf}
\caption { It can take several months, and even a few years, before a submitted
paper is officially published and citable. During this time, few people are
aware of the research that has been done (typically, close colleagues are
given access to the preprints). With public preprint servers, the science is
immediately available and can be openly discussed, analysed, and integrated
into current research. It benefits both science and publishers. Both want the
papers to be well-known and cited, and public preprints make it possible to
integrate research even before publication, greatly improving immediacy.  }
\label{fig:map}
\end{figure}

Preprint servers also establish priority in a fair way.  Some manuscripts will
spend much more time in the review process than others.  Public preprint servers
offer a much fairer way to establish intellectual priority by making the work
available when done, even if the exact organisation of the manuscript may
change. Surprisingly, there is a perception in biology that public preprints
make it easier to steal ideas, as if scientific ideas only took form in
published material.  Mathematicians and physicists have embraced arXiv in part
to establish priority in a fair way\cite{cal12}.

Some of the responses to public preprints are surprising since they are,
essentially, the same as exchanging preprints among colleagues.  Prepublication
reviews by a small network of colleagues is an important part of the scientific
process, which is attested by the fact that nearly all published papers
acknowledge comments by people not listed as co-authors.  Preprint servers
simply offer a way to extend this network of colleagues to the entire scientific
community. It ensures that science is not constrained by small networks of
scientists exchanging ideas.  Paul Ginsparg created arXiv.org in part for
democratic reasons: he wanted everyone from students in small universities to
Ivy-League professors to have access to the most recent scientific \emph{ideas}.
Ginsparg's revolutionary idea was simply to use the power of the internet for
preprints, not just for the end product, so the process can be open from A to Z,
instead of being just open at the end of the process.
% TP One important point to make: the MS and the idea are different things.  We
% should focus more on the later...

\section{Preprints, Ecology \& Evolution}

While submitting to public preprints servers is still uncommon in ecology and
evolution, preprints are becoming more common in biological sciences. The
quantitative biology section in arXiv is experiencing faster growth in
submissions than any other field \cite{cal12}. Also, most scientific journals
are preprint-friendly: Nature, PLOS, BMC, PNAS, Science (mostly)
\ref{table:policies}, and all the journals from Elsevier and Springer. Very
recently, the Ecological Society of America recently changed its policy to allow
public preprints (REF). In our field, few scientific publications will not
consider a manuscript submitted to arXiv.  Still, many ecology \& evolution
journals adopt a ``by default'' hostile attitude towards preprints, mostly due
to the lack of clear policy of the publishers. As an example, Wiley-Blackwell,
which publishes some of the leading journal in the field, has no official policy
on the subject \ref{table:policies}.

\begin{table*}
    \centering
    \begin{tabular}{|ll|}
    \hline
    Publisher                                   & Policy \\
    \hline
    Springer                            	& Accept \\
    BMC                                 	& Accept \\
    Elsevier                            	& Accept \\
    Nature Publishing Group             	& Accept \\
    Public Library of Science           	& Accept \\
    Royal Society                       	& Accept \\
    National Academy of Science (USA)           & Accept \\
    Ecological Society of America       	& Accept \\
    Oxford Journals                             & Accept \\
    Science                             	& Ambiguous \\
    Wiley-Blackwell                       	& No general policy \\
    British Ecological Society                  & No answer to our query \\
    \hline
    \end{tabular}
    \caption{Policies for important publishers in ecology and evolution.}
    \label{table:policies}
\end{table*}

There are other ``cultural'' barriers to adoption of preprint servers by
ecologists and evolutionary biologists.  The most notable preprint server,
arXiv, is designed to work best with \TeX{}, a nearly universal document
preparation system among physicists and mathematicians.  Jackson
\cite{jackson2002preprints} further argues that \TeX{} introduced an ``open
source mindset'' to its users, who now freely share their research findings as
well as their software.  Many biologists instead use proprietary software to
prepare their research findings, and many ecology and evolution journals
officially prefer Microsoft Word documents as submissions.  The recent
discipline-wide adoption of free software packages such as R and associated
document generation systems \cite{xie2012}, as well as interest in open access
publishing, has coincided with recent rise in use of preprint servers by
biologists, supporting Jackson's claims. \cite{xie12}

Physicists and other quantitative scientists have recently developed a
great interest in evolutionary theory, quantitative ecological theory
and epidemiological modeling.  Unfortunately this has led to a lot of
repeated work (``reinventing the wheel''; \cite{de2011contribution})
that could be avoided by better communication across disciplines.  The
near-universal adoption of preprint servers in physics provides this
vital communication channel.  We suggest that biologists reach out to
physicists and mathematicians by posting papers to arXiv: a physicist
who does not read \emph{Evolution} certainly checks arXiv at least
weekly.  This benefits both disciplines, as biologists will reach new
readers, and physicists will learn the terminology, tools and ideas
common in existing evolutionary theory.  A recent series of papers on
the theory of natural selection was posted to arXiv simultaneously
with its publication in the \emph{Journal of Evolutionary Biology}
\cite{JEB:JEB2431,JEB:JEB2498,JEB:JEB2378,JEB:JEB2373}.

\section{Current offerings}

We briefly discuss the main options to submit preprints to open servers:
arXiv.org, Figshare, and the upcoming PeerJ and F1000Research.

\subsection{arXiv}

% Rough outline:
% 1. Description of arXiv (origins, funding, maintenance)
% 2. Submission process
% 3. Role within math and physics community (nearly universal), including recent changes
% 4. Nature of citations and "priority"
%
% My biggest problem here is how to cite my sources: the information
% on funding comes from a press release from Cornell University
% Libraries; other basic "raw data" comes straight from the arXiv
% webpages. How do I cite that?
% -- Joel

arXiv (\url{http://arxiv.org/}) is the most widely-used preprint server today,
and its use is almost universal in mathematics and most branches of physics.
Physicist Paul Ginsparg originated arXiv in 1991 for theoretical high-energy
physicists to communicate preprints via email and ftp, and soon thereafter
adopted the newly created world-wide web\cite{jackson2002preprints}.  arXiv now
receives over 7,000 submissions per month
(\url{http://arxiv.org/show_monthly_submissions}).  arXiv divides its
submissions into subcategories of physics, mathematics, computer science,
quantitative biology, finance and statistics.  The quantitative biology category
includes subcategories for Populations and Evolution, Quantitative Methods and
other categories that may be of interest to ecologists and evolutionary
biologists.

Submission to arXiv is fully automated via the world-wide web.  Authors can
submit \TeX/\LaTeX documents that are compiled on the server, or directly submit
in PDF/PS format (for example, as exported by a word processor).  A moderation
system was put in place in 2004: papers must be categorized by an ``endorser.''
At least one author of a paper must be an endorser that has previously submitted
a paper or has received permission to submit to a particular category.  Many
authors in mathematics and physics submit papers as soon as they are ready for
review by colleagues, although another popular option is submitting
simultaneously to a journal and arXiv.

Most papers posted to arXiv are eventually printed in journals.  There are
notable exceptions, including Perelman's landmark paper leading to the proof of
the Poincar\'{e} conjecture \cite{2002math.....11159P}.  arXiv provides a
reliable citation system for all eprints (see our citation
\cite{2002math.....11159P}), lending a form of ``intellectual priority'' to
works posted there.  Despite these marks of arXiv functioning as a scientific
journal, arXiv has never sought to replace scientific journals and explicitly
states that it serves a different function as ``an openly accessible, moderated
repository for scholarly articles in specific scientific disciplines.''

arXiv is now administered by the Cornell University Libraries.  Funding comes
from voluntary pledges by academic institutions along with matching funds from
the Simons Foundation \cite{arxiv_future}.  One-hundred twenty six of the top
two-hundred institutions in terms of downloads have provided the operating
budget for arXiv over the next five years.  This plan reduces the financial
burden on Cornell University and transfers governance to a collaborative
community in accordance with arXiv's key principles.  arXiv takes numerous
measures to ensure that the repository will remain permanently available and
submissions will be readable.

\subsection{Figshare}

Figshare (\href{http://figshare.com/}{http://figshare.com/}) is an open server
that allows scientists to submit any research output: manuscript, figures,
datasets, videos, theses, presentations, and so on. There are no rules to limit
what constitutes a research output and, unlike arXiv, there is no endorser
system (although the administrator reserve the right to remove inappropriate
content). All figshare content (article, figures, datasets) have a unique
digital object identifier (DOI) like any journal article. Tags are used to
classify the content and new tags can be made. All content can be commented and
is licensed under the Creative Commons (CC-BY) license, except datasets which
are published under CC0.

\subsection{PeerJ}

% @Ethan

\subsection{F1000Research}

% Written from memory: to verify!

F1000Research is not a public preprint server like the previous three servers.
Whereas arXiv, Figshare, and PeerJ offer an option to submit a manuscript
without having it reviewed, papers submitted to F1000Research will eventually be
reviewed. Thus, F1000Research offers a hybrid model with publicly available
manuscripts at time of submission and standard peer-reviews. Manuscripts are
considered ``accepted'' and will only be indexed after two positive referee
response.

\section{Conclusion}

Responding to the rumour that they refused manuscripts submitted to arXiv,
Nature responded that ``Nature never wishes to stand in the way of communication
between researchers. We seek rather to add value for authors and the community
at large in our peer review, selection and editing'' \cite{nat05}.

Open preprints server offer a great opportunity for open science, especially if
the community embrace the idea of discussing preprints. Initiatives like
Haldane's Sieve (\href{http://haldanessieve.org/}{http://haldanessieve.org/}), a
new blog discussing arXiv papers in population genetics, will help make arXiv
attractive for scientists looking to promote their work. These initiatives are
important to fully exploit the potential of open preprints servers.

Posting preprints online increases the community of available informal peer
reviewers, and uses the internet for its original community-building purposes.
Preprint servers also facilitate communication between disciplines, bridging
cultural as well as geographic divides. The advantages are clear and the costs are low.

% Short list of disadvantages

\newpage
\bibliography{refs}
\bibliographystyle{plain}

\end{document}

