\documentclass[letterpaper,twocolumn,superscriptaddress,showkeys]{revtex4}
\usepackage[utf8]{inputenc}
\usepackage{color,dcolumn,graphicx,hyperref}
\hypersetup
{
    colorlinks = true, linkcolor = blue, citecolor = blue, urlcolor = blue,
}

\begin{document}

\title{Open Preprints in Ecology \& Evolution}

\author{Philippe Desjardins-Proulx}
\email[E-mail: ]{philippe.d.proulx@gmail.com}
\affiliation{Theoretical Ecosystem Ecology laboratory, Universit\'e du Qu\'ebec \`a Rimouski, Canada.}
\affiliation{Quebec Center for Biodiversity Science, McGill University, Canada.}
\affiliation{D\'epartement des sciences biologiques, Universit\'e du Qu\'ebec \`a Montr\'eal, Canada.}

\author{Ethan P. White}
\affiliation{Departement of Bology, Utah State University, United-States of America.}

\author{Timoth\'ee Poisot}
\affiliation{Theoretical Ecosystem Ecology laboratory, Universit\'e du Qu\'ebec \`a Rimouski, Canada.}
\affiliation{Quebec Center for Biodiversity Science, McGill University, Canada.}
\affiliation{International Network for Next-Generation Ecology.}

\author{Dominique Gravel}
\affiliation{Theoretical Ecosystem Ecology laboratory, Universit\'e du Qu\'ebec \`a Rimouski, Canada.}
\affiliation{Quebec Center for Biodiversity Science, McGill University, Canada.}

\author{Karthik Ram}
\affiliation{Environmental Science, Policy, and Management. University of California, Berkeley. Berkeley, CA. United-States of America.}



\keywords{Publishing; arXiv; Green Open Access.}

\begin{abstract}

...
 
\end{abstract}

\maketitle

\section{Introduction}

Open preprints servers allow authors to make their manuscripts publicly available before, or in parallel to, submitting them to journals for traditional
peer-review. This idea gained popularity 20 years ago with the advent of arXiv, an open preprint server, widely used in the physical sciences \cite{gin11}. The idea behind pre-print servers is fundamentally simple: to make the results of a scientific endavour available to the scientific community as soon as possible rather than wait until the peer-review process is fully completed. The point of arXiv and open preprints servers is not to avoid the peer-review. Almost all manuscripts submitted to arXiv undergo formal peer-review at a journal. The point is to open an important phase of the publication cycle ...

In this article, we will first highlight the advantages of open preprints
servers for both scientists and publishers. We will also debunk a few
misconceptions, discuss the policies of major publishers in ecology an
evolution, and briefly review the most popular open preprint servers.

\section{The case for open preprints}

The first and most often discussed advantage of arXiv and open preprints is
speed \ref{fig:map}. The time between submission and the official publication of
a manuscript can be measured in months, sometime in years. For all this time,
the research is only known to a select few: colleagues, editors, reviewers.
Thus, the science cannot be used, discussed, or reviewed by the wider scientific
community. ...

%% Allows papers to be cited earlier (greater immediacy)

% Tim's section:
The review process as a whole is critically over-loaded, because the number of
active scientists increases, because the pressure to publish increases, and
because of an effect dubbed ``the tragedy of the reviewers commons'' REF.  In
the same times, rejection rates are high in most journals (REF), and when the
not invited to submit a revisions, authors are left with the impression that
they must start the whole process all over again. It's thus no surprise that
different initiatives emerged over the last few years, to decrease the time
spent in review. XXX et coll. (REF) called for the recycling and reuse of
peer-reviews: by attaching previous reviews, and detailed replies, to a new
submission, both the editor and the referees can gauge the work done on the
manuscript, and perhaps evaluate it with less prejudice. In a similar way, the
\emph{Peerage of Science} initiative allows authors to seek anonymous pre-
review by their peers. Some journals (LIST?) now accept to publish papers
which received good evaluations, effectively outsourcing the review process. A
widespread use of preprint servers can achieve the same goal of reducing the
time spent in review. By putting a manuscript out there for open comments and
criticisms, the authors will receive valuable feedback, and can improve the
version which will be submitted. With a rich enough community of scientists
depositing preprints, and commenting on them, the process of an open pre-
review can become widespread, and will overall increase the quality of first
submissions.

\begin{figure}[ht!] \centering\includegraphics[width=0.50\textwidth]{map.pdf}
\caption { It can take several months, and even a few years, before a submitted
paper is officially published and citable. During this time, few people are
aware of the research that has been done (typically, close colleagues are
given access to the preprints). With public preprint servers, the science is
immediately available and can be openly discussed, analysed, and integrated
into current research. It benefits both science and publishers. Both want the
papers to be well-known and cited, and public preprints make it possible to
integrate research even before publication, greatly improving immediacy.  }
\label{fig:map}
\end{figure}

% @to myself: rewrite this paragraph:
Some of the responses to public preprints are surprising since they are,
essentially, the same as exchanging preprints among colleagues.
% TP Well, no. You trust your colleagues, and you trust that they will not give your
%    MS to anyone!
Prepublication reviews by a small network of colleagues is an important part
of the scientific process, which is attested by the fact that nearly all
published papers acknowledge comments by people not listed as co-authors.
Preprints servers simply offer a way to extend this network of colleagues to
the entire scientific community. It ensures that science is not constrained by
small networks of scientists exchanging ideas. Ginsparg made arXiv.org in part
for democratic reasons: he wanted everyone from graduate students in small
universities to Princeton professors to have access to the most recent
scientific \emph{ideas}.
% TP One important point to make: the MS and the idea are different things.
%    We should focus more on the later...
Ginsparg revolutionary idea was simply to use the power of the internet for
preprints, not just for the end product, so the process can be open from A to
Z, instead of being just open at the end of the process (and often not open at
all).
% TP Since when is the publishing buisness an open process? :-)
% PhDP: Wishful thinking :) Corrected...

Preprint servers also establish priority in a fair way. Since some manuscripts
will spend much more time in the review process, public preprints servers offer
a fairer way to establish intellectual priority by making the work available
when done. Surprisingly, there is perception in biology that public preprints
make it easier to steal ideas, while mathematicians and physicists have embraced
arXiv in part to establish intellectual priority in a fair way \cite{cal12}.

Preprint servers also provide additional opportunities for review, often from members outside the list of authors or their immediate peers (cite). This process can further improve the quality of a manuscript before it reaches potential reviewers. pre-prints servers can therefore reduce reviewer load/burden by opening up oppportunities for additional feedback.
 

% Karthik's section
% I'll need to flesh this section out a bit more but here are the two main points
% pre-print servers enrich the review process by providing a venue to solicit comments from anyone. Manuscripts can undergo several rounds of revision during this stage, resulting in a higher quality manuscrpt reaching reviewers than ones that have not undergone this process. There are various benefits here. First, it can reduce the time burden on reviewers, the need for multile rounds of revision.

% Second point, pre-publication provides a scaffolding the possibility of a implementing a reward system for reviewers and publishers. Open reviews can be judged on their quality and usefulness both by the authors and by other members of the community. Reviews and comments deemed useful by the community can be used to reward contributers with reputation, which then provides a measure of trustworthiness. The reputation system can also serve to deter malicious reviews and personal attacks, and such penalties can limit  @PhDP: Do you think this is a useful direction to go? It could be moved into the discussion or stay with the benefits of pre-print servers.

\section{Preprints, Ecology \& Evolution}

While submitting to public preprints servers is still uncommon in ecology and
evolution, preprints are becoming more common in biological sciences. The
quantitative biology section in arXiv is experiencing faster growth in
submissions than any other fields \cite{cal12}. Also, most scientific journals
are preprint-friendly: Nature, PLOS, BMC, PNAS, Science (mostly)
\ref{table:policies}, and all the journals from Elsevier and Springer. Very
recently, the Ecological Society of America amended its policy to allow
public preprints (REF). In our field, few scientific publications will not
consider a manuscript submitted to arXiv.  Still, many ecology \& evolution
journals adopt a ``by default'' hostile attitude towards preprints, mostly due
to the lack of clear policy of the publishers. As an example, Wiley-Blackwell,
which publishes some of the leading journal in the field, has no official policy
on the subject \ref{table:policies}.

\begin{table*}
    \centering
    \begin{tabular}{|ll|}
    \hline
    Publisher                                   & Policy \\
    \hline
    Springer                            	& Accept \\
    BMC                                 	& Accept \\
    Elsevier                            	& Accept \\
    Nature Publishing Group             	& Accept \\
    Public Library of Science           	& Accept \\
    Royal Society                       	& Accept \\
    National Academy of Science (USA)           & Accept \\
    Ecological Society of America       	& Accept \\ % Victory!
    Science                             	& Accept/Ambiguous \\
    Wiley-Blackwell                       	& No general policy \\ % grrrrrr
    British Ecological Society                  & ? \\ % Waiting for their answer...
    % Oxford worth mentionning?
    \hline
    \end{tabular}
    \caption{Policies for important publishers in ecology and evolution.}
    \label{table:policies}
\end{table*}

\section{Current offer}

We briefly discuss the main options to submit preprints to open servers:
arXiv.org, Figshare, and the upcoming PeerJ and F1000Research.

\subsection{arXiv}

arXiv (\href{http://arxiv.org/}{http://arxiv.org/}).
arXiv is funded by a network of universities.

...

% @Joel

\subsection{Figshare}

% @PhDP

Figshare (\href{http://figshare.com/}{http://figshare.com/}) is an open servers
that allow scientists to submit any research output: manuscript, figures,
datasets, videos, theses, presentations, and so on. There are no rules to limit
what constitutes a research output: anything

All figshare content (article, figures, datasets) have a unique digital object
identifier (DOI) like any journal article.

All content is licensed under the Creative Commons (CC-BY) license, except
datasets which are published under CC0.
% Small explanation

\subsection{PeerJ}

% @Ethan

\subsection{F1000Research}

% Written from memory: to verify!

F1000Research is not a public preprint server like the previous three servers.
Whereas arXiv, Figshare, and PeerJ offer an option to submit a manuscript
without having it reviewed, papers submitted to F1000Research will eventually be
reviewed. Thus, F1000Research offers a hybrid model with publicly available
manuscripts at time of submission and standard peer-reviews. Manuscripts are
considered ``accepted'' and will only be indexed after two positive referee
response.

\section{Conclusion}

% Poorly written:
Responding to the rumour that they refused manuscripts submitted to arXiv,
Nature responded that ``Nature never wishes to stand in the way of communication
between researchers. We seek rather to add value for authors and the community
at large in our peer review, selection and editing'' \cite{nat05}.

Open preprints server offer a great opportunity for open science, especially if
the community embrace the idea of discussing preprints. Initiatives like
Haldane's Sieve (\href{http://haldanessieve.org/}{http://haldanessieve.org/}), a
new blog discussing arXiv papers in population genetics, will help make arXiv
attractive for scientists looking to promote their work. These initiatives are
important to fully exploit the potential of open preprints servers.

\newpage
\bibliography{refs}
\bibliographystyle{plain}

\end{document}

